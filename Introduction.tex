\section{Introduction}
	\subsection{Problem Background}
	As an indespensable part of our terrestrial ecosystem, fungi free the carbon and other elements out from remains and debris and drive them into the ecosystem. Fungi tend to live in warm and humid environment,and are sensitive with the smallest changes.

	In this study, we focus on the interaction between different population of fungi and how they interact with microenvironment around them on woddy fibres. We use cmpetitive Lotka-Voterra model to demostrate the competition among different types of fungi.
	\subsection{Restatement of the Problem}
	\begin{enumerate}[\bfseries 1.]
	\item Build a mathematical model to desccribe the process of multiple populatioons of fungi breaking ground litter  without competition. The biomass of each kind of fungi is the function of temperature and moisture in their microenvironemnt.
	\item Take competition into consideration, and modify the model to study the short-term and long-term behavior of fungi populations. Show how fungi compete under the influence of temperature, moisture and 
	\item Consider actual data of temperature, moisture ,decomposition and competition and give the best living conditions of each kind of fungi.
	\item Give the law how population diversity affects the decomposition efficiency, and estimate the importance of biodiversity in different variability of environment.
	\end{enumerate}
	\subsection{Our Work}
	Our work mainly includes Sthe following:
	\begin{enumerate}[\bfseries 1.]
		\item We use Lotka-Volterra Equations to build a model to demonstrate the competition among different types of fungi.
		\item We build a model to explain how the process of decomposition is affected by temperature, moisture, and distribution of different tuypes of fungi.
		\item We analyse the sensitivity and robustness of the models above.
	\end{enumerate}
	